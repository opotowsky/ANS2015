\documentclass{anstrans}
\usepackage[acronym, toc]{glossaries}
\newacronym{MIT}{MIT}{the Massachusetts Institute of Technology}
\newacronym{UW}{UW}{University of Wisconsin}
\newacronym{US}{US}{United States}
\newacronym{IAEA}{IAEA}{International Atomic Energy Agency}
\newacronym{SNF}{SNF}{spent nuclear fuel}
\newacronym{HLW}{HLW}{high level waste}
\newacronym{FEHM}{FEHM}{Finite Element Heat and Mass Transfer}
\newacronym{DOE}{DOE}{Department of Energy}
\newacronym{DHS}{DHS}{Department of Homeland Security}
\newacronym{GENIUSv1}{GENIUS}{Global Evaluation of Nuclear Infrastructure 
Utilization Scenarios, Version 1}
\newacronym{GENIUSv2}{GENIUS}{Global Evaluation of Nuclear Infrastructure Utilization Scenarios, Version 2}
\newacronym{CNERG}{CNERG}{Computational Nuclear Engineering Research Group}
\newacronym{GDSM}{GDSM}{Generic Disposal System Model}
\newacronym{GDSE}{GDSE}{Generic Disposal System Environment}
\newacronym{GPAM}{GPAM}{Generic Performance Assessment Model}
\newacronym{FEPs}{FEPs}{Features, Events, and Processes}
\newacronym{EBS}{EBS}{Engineered Barrier System}
\newacronym{EDZ}{EDZ}{Excavation Disturbed Zone}
\newacronym{YMR}{YMR}{Yucca Mountain Repository Site}
\newacronym{EPA}{EPA}{Environmental Protection Agency}
\newacronym{PEI}{PEI}{Peak Environmental Impact}
\newacronym{VISION}{VISION}{the Verifiable Fuel Cycle Simulation Model}
\newacronym{NUWASTE}{NUWASTE}{Nuclear Waste Assessment System for Technical Evaluation}
\newacronym{NWTRB}{NWTRB}{Nuclear Waste Technical Review Board}
\newacronym{OCRWM}{OCRWM}{Office of Civilian Radioactive Waste Management}
\newacronym{UFD}{UFD}{Used Fuel Disposition}
\newacronym{DYMOND}{DYMOND}{Dynamic Model of Nuclear Development }
\newacronym{DANESS}{DANESS}{Dynamic Analysis of Nuclear Energy System Strategies}
\newacronym{CAFCA}{CAFCA}{Code for Advanced Fuel Cycles Assessment }
\newacronym{ORION}{ORION}{O..}
\newacronym{NFCSim}{NFCSim}{Nuclear Fuel Cycle Simulator}
\newacronym{COSI}{COSI}{Commelini-Sicard}
\newacronym{FCT}{FCT}{Fuel Cycle Technology}
\newacronym{SWF}{SWF}{Separations and Waste Forms}
\newacronym{FCO}{FCO}{Fuel Cycle Options}
\newacronym{RDD}{RD\&D}{Research Development and Design}
\newacronym{WIPP}{WIPP}{Waste Isolation Pilot Plant}
\newacronym{ANDRA}{ANDRA}{Agence Nationale pour la gestion des D\'echets RAdioactifs, the French National Agency for Radioactive Waste Management}
\newacronym{TSM}{TSM}{Total System Model}
\newacronym{LANL}{LANL}{Los Alamos National Laboratory}
\newacronym{INL}{INL}{Idaho National Laboratory}
\newacronym{ANL}{ANL}{Argonne National Laboratory}
\newacronym{SNL}{SNL}{Sandia National Laboratory}
\newacronym{LBNL}{LBNL}{Lawrence Berkeley National Laboratory}
\newacronym{LLNL}{LLNL}{Lawrence Livermore National Laboratory}
\newacronym{NAGRA}{NAGRA}{National Cooperative for the Disposal of Radioactive Waste}
\newacronym{CUBIT}{CUBIT}{CUBIT Geometry and Mesh Generation Toolkit}
\newacronym{CSNF}{CSNF}{Commercial Spent Nuclear Fuel}
\newacronym{DSNF}{DSNF}{DOE Spent Nuclear Fuel}
\newacronym{MTHM}{MTHM}{Metric Ton of Heavy Metal}
\newacronym{HTGR}{HTGR}{High Temperature Gas Reactor}
\newacronym{TRISO}{TRISO}{Tristructural Isotropic}
\newacronym{MA}{MA}{Minor Actinide}
\newacronym{CEA}{CEA}{Commissariat a l'Energie Atomique et aux Energies Alternatives}
\newacronym{SKB}{SKB}{Svensk Karnbranslehantering AB}
\newacronym{SINDAG}{SINDA{\textbackslash}G}{Systems Improved Numerical Differencing Analyzer $\backslash$ Gaski}
\newacronym{STC}{STC}{Specific Temperature Change}
\newacronym{LDRD}{LDRD}{Laboratory Directed Research and Development}
\newacronym{LCOE}{LCOE}{Levelized Cost of Electricity}
\newacronym{ABM}{ABM}{Agent-Based Modeling}
\newacronym{COTS}{COTS}{Commercial, Off-The-Shelf}
\newacronym{API}{API}{Application Programming Interface}
\newacronym{RIF}{RIF}{Region-Institution-Facility}
\newacronym{GUI}{GUI}{Graphical User Interface}
\newacronym{HPC}{HPC}{High-Performance Computing}
\newacronym{HTC}{HTC}{High-Throughput Computing}
\newacronym{UML}{UML}{Unified Modeling Language}
\newacronym{MOX}{MOX}{Mixed Oxide}
\newacronym{UOX}{UOX}{Uranium Oxide}
\newacronym{QA}{QA}{Quality Assurance}
\newacronym{NQA1}{NQA-1}{Nuclear Quality Assurance - 1}
\newacronym{VV}{V\&V}{Verification and Validation}
\newacronym{UQ}{UQ}{Uncertainty Quantification}
\newacronym{ASME}{ASME}{American Society of Mechanical Engineers}
\newacronym{NEAMS}{NEAMS}{Nuclear Engineering Advanced Modeling and Simulation}
\newacronym{CI}{CI}{Continuous Integration}
\newacronym{DAG}{DAG}{Directed Acyclic Graph}
\newacronym{XML}{XML}{Extensible Markup Language}
\newacronym{RNG}{RelaxNG}{REgular LAnguage for XML Next Generation}
\newacronym{JSON}{JSON}{JavaScript Object Notation}
\newacronym{SQL}{SQL}{Structured Query Language}
\newacronym{SQLite}{SQLite}{Structured Query Lite}
\newacronym{HDF5}{HDF5}{Hierarchical Data Format version 5}
\newacronym{CSV}{CSV}{Comma-Separated Value}
%\newacronym{<++>}{<++>}{<++>}

\makeglossaries
%%%%%%%%%%%%%%%%%%%%%%%%%%%%%%%%%%%
\title{Cymetric: Fuel Cycle Metrics using \cyclus}
\author{Anthony Scopatz, Arrielle Opotowsky, Paul P. H. Wilson$^{*}$}

\institute{
$^{*}$Department of Engineering Physics, University of Wisconsin - Madison, 
1500 Engineering Drive, Madison WI 53703
}

\email{scopatz@gmail.edu \and opotowsky@wisc.edu}

%%%% packages and definitions
\usepackage{graphicx} % allows inclusion of graphics
\usepackage{booktabs} % nice rules (thick lines) for tables
\usepackage{microtype} % improves typography for PDF
\usepackage{xspace}
\usepackage{float}
\usepackage{ulem}

\newcommand{\SN}{S$_N$}
\newcommand{\cyclus}{\textsc{Cyclus}\xspace}
\newcommand{\TODO}[1] {{\color{red}\textbf{TODO: #1}}}

%%%% code block stuff
\usepackage{listings}
\usepackage{textcomp}
\usepackage{color}

\definecolor{code}{rgb}{0,0.25,0.25}
\lstset{
    language={Python},
    tabsize=4,
    rulecolor=\color{black},
    upquote=true,
    aboveskip={1.5\baselineskip},
    belowskip={1.5\baselineskip},
    columns=fixed,
    extendedchars=true,
    breaklines=true,
    prebreak=\raisebox{0ex}[0ex][0ex]{\ensuremath{\hookleftarrow}},
    frame=single,
    showtabs=false,
    showspaces=false,
    showstringspaces=false,
    basicstyle=\scriptsize\ttfamily\color{code},
    keywordstyle=\color[rgb]{0,0,1.0},
    commentstyle=\color[rgb]{0.133,0.545,0.133},
    stringstyle=\color[rgb]{0.627,0.126,0.941},
    numberstyle=\color[rgb]{0,1,0},
    identifierstyle=\color{code},
    captionpos=t,
}
\newcommand{\code}[1]{{\color{code}\texttt{#1}}}
    
\begin{document}
%%%%%%%%%%%%%%%%%%%%%%%%%%%%%%%%%%%%%%%%%%%%%%%%%%%%%%%%%%%%%%%%%%%%%%%%%%%%%%%%
\section{Introduction}
Nuclear fuel cycle performance is being given more attention as energy experts are attempting to identify a path forward for \gls{US} nuclear power. Thus, fuel cycle simulators are of interest; they can provide rapid feedback on an array of nuclear fuel cycles. This is generally done by computing metrics from the simulation results. Metrics of interest to the \gls{US} \gls{DOE}, for example, include levelized cost of electricity, proliferation resistance, and various quantities and qualities of the \gls{SNF}.

The \cyclus nuclear fuel cycle simulator is a unique approach to the suite of existing codes. It includes many benefits and capabilities, including a modular architecture for flexible simulations and extensive comparisons, as well as an agent-based modeling approach to improve upon traditional system dynamics. \TODO{hrm, not sure what to cite, and not sure if I should add more than a blurb like this}

Because a simulation is typically evaluated and compared via metrics, an analysis tool was written for \cyclus : Cymetric. When a \cyclus simulation is complete, it outputs a database. Cymetric reads in data from this \cyclus -generated database, then computes and writes the calculated metrics back to the database to allow for retrieval. The metrics can be dependent on a column in a table in a \cyclus database or on another metric.

Cymetric was designed with usability and extensibility in mind; it has a Python interface and enables user customizations and input. Cymetric's simple implementation  features a command line utility, \code{cymetric}, as well as Python functionality. It is also designed so that a user can easily add their own metrics of interest if they are not already included in the tool. 

\TODO{more in intro}

%%%%%%%%%%%%%%%%%%%%%%%%%%%%%%%%%%%%%%%%%%%%%%%%%%%%%%%%%%%%%%%%%%%%%%%%%%%%%%%%
\section{Design}
\TODO{ read in from db, write to db, define root tables}

Cymetric is designed to allow for metrics to be dependent on other metrics. For example, one might be interested in the decay heat of a batch of \gls{SNF}. The decay heat is calculated using the activity of individual nuclides. The activity is calculated from the mass of each nuclide, which is calculated from two root tables in the \cyclus database: quantity of material (mass) and the mass fraction of each nuclide in that material set. If the activity was previously calculated for another purpose, it is stored in the database and the additional step of determining the decay heat will be less expensive. 

\TODO{I could make a figure to describe this if you think I should}

%%%%%%%%%%%%%%%%%%%%%%%%%%%%%%%%%%%%%%%%%%%%%%%%%%%%%%%%%%%%%%%%%%%%%%%%%%%%%%%%
\section{Implementation}

\subsection{Cymetric via the Command Line}
The simplest way to become familiarized with the functionality of Cymetric is through the command line utility, accessed via \code{cymetric}. 

The first argument given is the database of interest. Currently, \cyclus supports \gls{HDF5} and \gls{SQLite} databases, so Cymetric supports both as well. For the purposes of this paper, \code{test.sqlite} will be used as the cyclus output. The second argument given is a flag: either \code{-l} to obtain a listing of the tables in the database or \code{-e} to execute arbitrary code on a root table or a metric. 

A table listing using \code{-l} will print the \cyclus interface tables as well as the metadata \cyclus stores, as shown in Listing \ref{dashl}. After a metric is computed, it shows up on the table listing as well. 
\begin{lstlisting}[caption ={List of Tables in a Database}, label=dashl]
$ cymetric test.sqlite -l
AgentEntry
AgentStateAgent
AgentStateInventories
AgentState_agents_NullInstInfo
AgentState_agents_NullRegionInfo
AgentState_cycamore_BatchReactorCommodPrefs
AgentState_cycamore_BatchReactorCommodityRecipeContext_inoutmap
AgentState_cycamore_BatchReactorCommodityRecipeContext_resmap
AgentState_cycamore_BatchReactorInfo
AgentState_cycamore_BatchReactorInitialInv
AgentState_cycamore_BatchReactorRecipeChanges
AgentState_cycamore_EnrichmentFacilityInfo
Compositions
DecayMode
Enrichments
FieldTypes
Finish
Info
InputFiles
MapIntDouble
MapIntInt
MapIntStr
MapStrDouble
MapStrInt
MapStrStr
MaterialInfo
NextIds
Prototypes
Recipes
ResCreators
Resources
Snapshots
Transactions
VectorDbl
VectorInt
VectorStr
XMLPPInfo
\end{lstlisting}

The code execution using \code{-e} is accomplished by including Python code in quotes after the flag, as shown in Listing \ref{dashe}. 
\begin{lstlisting}[caption ={Executing Code on a Database}, label=dashe]
$ cymetric test.sqlite -e "PYTHON CODE HERE"
\end{lstlisting}

When a metric or root table is used by Cymetric, it returns the table as a pandas data frame \TODO{cite}. If the table Resources was of interest, then one enters \code{print(Resources[:])} inside the quotation marks. This outputs the table, a portion of which is shown below in Table \ref{tab:resources}. The full metric is returned if given \code{[:]}, \code{[None]}, or \code{[...]}. Otherwise, the metrics can be filtered.

\begin{table*}[htb]
  \centering
\begin{tabular}{llllllllll}
      &       & SimId      & ResourceId      &
      ObjId      & Type      & TimeCreated      & Quantity      &
      Parent1      & Parent2
\\
\end{tabular}
  \caption{The Resources table in the \cyclus database.}
  \label{tab:resources}
\end{table*}

Condition filters on column names can be applied and separated by commas if there is more than one. For example, if the plutonium isotopes are of interest for an analysis, they can be isolated as shown in Listing \ref{pu}. 

\begin{lstlisting}[caption ={Filtering a Root Table in a Database}, label=pu]
 $ cymetric test.sqlite -e "print(Compositions[NucId >= 940000000, NucId < 950000000])"
      SimId     QualId      NucId  MassFrac
0  62079c87...       5  942380000  0.000269
1  62079c87...       5  942390000  0.005667
2  62079c87...       5  942400000  0.002204
3  62079c87...       5  942410000  0.001944
4  62079c87...       5  942420000  0.000820

[5 rows x 4 columns]
\end{lstlisting}

In addition to filters on root tables, metrics can  be calculated directly on the command line. A predefined metric, \code{Materials}, can be called to calculate the mass of each nuclide for each transaction made during the simulation. Additionally, if the masses are in $g$ yet $kg$ is needed, a conversion factor can be applied, as shown in Listing \ref{mass}.

\begin{lstlisting}[caption ={Calculating and Manipulating Metrics}, label=mass]
$ cymetric test.sqlite -e "mass = Materials[:]['Mass'] / 1000; print(mass)"
0      0.711000
1     99.289000
2      0.710954
3     99.282530
4      0.000046
5      0.006470
6      0.710907
7     99.276059
8      0.000046
9      0.006470
10     0.710827
11    99.264806
12     0.000081
13     0.011254
14     0.710790
...
Name: Mass, Length: 209, dtype: float64
\end{lstlisting}

To assist more than just printing tables, there are a number of Python modules available when using the execution  flag. These are listed below in Table \ref{tab:modules} with their aliases.

\begin{table}[htb]
  \centering
\begin{tabular}{ll}
\toprule
  Module                   & Alias \\
\midrule
  \code{cymetric}          & \code {cym} \\
  \code{numpy}             & \code{np} \\  
  \code{pandas}            & \code{pd} \\
  \code{uuid}              & \code{uuid} \\
  \code{matplotlib}        & \code{matplotlib} \\
  \code{matplotlib.pyplot} & \code{plt} \\
\bottomrule
\end{tabular}
  \caption{Modules available via Cymetric command line code execution.}
  \label{tab:modules}
\end{table}

\TODO{Talk about getting a pretty graph using matplotlib}

\subsection{Cymetric via Python}
To obtain some degree of automation, Cymetric is accessible via Python as well. It is recommended that the aliases in \ref{tab:modules} be used. The same functionality as described above is available using the functions described below in Table \ref{tab:pyfunc}.

\begin{table}[htb]
  \centering
\begin{tabular}{ll}
\toprule
  Function           & Syntax                 \\
\midrule 
  \code{dbopen()}    & \code{dbopen(`db')}         \\
  \code{eval()}      & \code{eval(`Metric', db)}   \\
  \code{exec\_code()} & \code{exec\_code(``PYTHON CODE'', db)}  \\ 
  \code{root\_metric()} & \code{root\_metric(name=`CustomTable')}  \\
\bottomrule
\end{tabular}
  \caption{Cymetric capabilities provided as Python functions.}
  \label{tab:pyfunc}
\end{table} 

An example of an implementation of the above functions is shown below in Listing \ref{funcs}. The database is opened via \code{dbopen()}. The function \code{eval()} evaluates and/or reads a metric or root table from the database. It allows for filtering conditions in 3-string tuples; more filters can be added by separating the tuples with commas. Lastly, \code{exec\_code()} provides the same functionality as \code{cymetric} on the command line. \TODO{I'm not perfectly clear on the advantage here}

\begin{lstlisting}[caption ={Example Python Script Using Cymetric}, label=funcs]
import cymetric as cym

db = cym.dbopen('test.sqlite')

# print 
evaler = cym.Evaluator(db)
frame = evaler.eval('Materials', db, conds=[('NucId', '==', '922350000')])
print(frame)

# accomplishes the same output as above
cym.exec_code("print(Materials[NucId == 922350000])", db)
\end{lstlisting}

In Listing \ref{funcs}, an \code{Evaluator} object was created because \code{eval()} creates a new object every time it is called. In the spirit of efficiency, creating an object in this way and calling its method directly prevents the database from being read each time \code{eval()} is invoked. 

Lastly, \cyclus allows for the creation of custom tables, so these are not accessible in Cymetric as the default root tables are. Therefore, a custom table can be accessed by the \code{root\_metric()} function. \TODO{Should I also show a code example here?}

\subsection{Writing Metrics in Cymetric}
The richness of Cymetric will be largely from user contributions to metrics. For example, radiotoxicity from \gls{SNF} can be calculated in many different ways. Sieverts per generated electricity or Sieverts per metric ton of uranium or \gls{SNF} are typically used to quantify this value. Additionally, the doses are determined differently; ingestion dose rates or total effective equivalent doses may be used. Since the developers cannot cover all scenarios, user-driven extensibility was built into Cymetric. 

A metric must be written with a function that accepts a pandas series and returns a pandas data frame. It also has the \code{$@$metric()} decorator, which takes three arguments: the name of the metric, its dependencies, and the metric's structure, or schema. The name is given as a string. The dependencies are a list of 3-entry tuples. Each one includes the name of the table, a tuple column names that are to index the series, and the column that is the value of the series. The schema provides the structure of the metric table and is a list of 2-entry tuples: the column name followed by its \cyclus datatype. A simple example is depicted below in Listing \ref{massmetric}. This metric is only dependent on the column Mass from the Materials metric. 

\begin{lstlisting}[caption ={Writing a Metric in Cymetric}, label=massmetric]
    deps = [('Materials', ('SimId', 'ResourceId', 'NucId'), 'Mass')]

    schema = [('SimId', cym.UUID), ('ResourceId', cym.INT),
              ('NucId', cym.INT),  ('MassSquared', cym.DOUBLE)]

    @cym.metric(name='MaterialsSquared', depends=deps, schema=schema)
    def mats_sqrd(series):
        mats = series[0]
        rtn = mats**2
        rtn.name = 'MaterialsSquared'
        rtn = rtn.reset_index()
        return rtn
\end{lstlisting}


%%%%%%%%%%%%%%%%%%%%%%%%%%%%%%%%%%%%%%%%%%%%%%%%%%%%%%%%%%%%%%%%%%%%%%%%%%%%%%%%
\section{Conclusions}

I'll write a conclusion soon. 

\section{Acknowledgments}
This work has been funded by the \gls{DOE} Nuclear Energy University Program and the \gls{DHS} Nuclear Forensics Graduate Research Fellowship Program. 

%%%%%%%%%%%%%%%%%%%%%%%%%%%%%%%%%%%%%%%%%%%%%%%%%%%%%%%%%%%%%%%%%%%%%%%%%%%%%%%%
\bibliographystyle{ans}
\bibliography{cymetric}
\end{document}